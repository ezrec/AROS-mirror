\documentstyle[12pt]{book}

%\textwidth 6in
%\textheight 8in
%\parindent .5cm
%\voffset -.75in
%\hoffset -.5in
%\makeindex
\def\version{ V2.00 }
\input XDMEMacros

\begin{document}

\title{XDME \version}
\author{Aaron Digulla\\Dirk Heckmann\\Bernd Noll}
\maketitle

\tableofcontents

\chapter{Introduction}\index{Introduction}

\section{DME}\index{DME}

This is version \version\ of XDME; an ``update'' to the popular DME by Matthew
Dillon. Dillon did no further work on DME since V1.45 (because he's too
busy) and so I asked him whether I might do it. I have done much work on
the source and on the handling and hope you'll like it.

Although I tried not to change too much, I had to remove several commands
and did some changes to the handling that are incompatible to the old DME.
I'm sorry for this, but I think that the advantages of the changes override
the disadvantage that you have to go through your old scripts.

\section{Copyright}\index{Copyright}

DME � Copyright 1987-91, Matthew Dillon.  All Rights Reserved. \\
XDME � Copyright 1992-93, Aaron Digulla. All Rights Reserved.

\section{License}\index{License}

Oh well, as always

\begin{itemize}
\item Don't make profit out of it and don't change the copyright. You can change
    anything in this package, tough, as long as you make a clear statement
    what you did change, why and who you are. I don't like being made
    responsible for somebody else's mistakes.
\item You can make as many copies of this as you like. But you must not make
    profit out of it.
\item This software is to help people, so the usage of this software in
    organizations that do dubious things like weapons and health-penetrating
    chemicals is STRICTLY FORBIDDEN !
\item I don't give any warranty of any kind. So if you find errors, you
    cannot punish me for it, but if you tell me about it, I'll fix it
    as soon as possible. I like error-free software, too :-)
\end{itemize}

\section{Overview}\index{overview}\index{history}

M. Dillon released his DME in 1987 and many programmers liked it first
view. This was due to the fact that DME was very fast and powerful and the
only other editor was ed :-). Dillon enhanced and improved his editor
several times until 1991 when he began his work on DICE. From this point,
DME wasn't improved anymore although there were many bugs and problems
pending. I did some bug-fixes and programmed several new commands. When I
sent this package to Dillon, he told me that he won't support DME anymore
and I could go on with it if I want, but I do have to change the name. I
decided to name it XDME, because (1) it is an enhancement of DME and (2)
I'm working on a X11-Version\index{X11-version}\index{X11DME} for
UNIX\index{UNIX} :-). I don't like VI very much :-))

Since then much happened. There are many new commands, several speed- and
memory-improvements have been made and we gave it a more powerful
AREXX\index{AREXX} and programming-language. Right now, three people work
on it: Me, Dirk Heckmann and B. Noll. I had the initial idea and began the
work, Dirk made the AREXX-stuff and B. Noll sent me many new commands. Not
to forget many Beta-testers that risked their health with testing and using
this :-)

XDME is a really powerful tool although it has no real WYSIWYG like in any
DTP-Program. XDME's power is to edit raw ASCII-Text and to automate the work.
With over 200 commands at your fingertips, XDME is surely the most
powerful text-editor ever. It has been designed to be fast and usable and
to aid programmers in the task of programming. It has almost any command
you can think of and if it has not, then you can create it !

\subsection{The Window}\index{main window}\index{main display}

When you open a new XDME, you will see a new window appearing. This window
has three significant parts. First the main part is occupied by a large area
to edit text. This is why we call this thing an ``text-editor'' :-). The next
vital thing is the status-line which is placed in the window-title. This is
to give you one more line of text in the main area\index{text area} ! WOW !
This status line\index{status line} shows several information like your
actual position, the number of lines the text has, the character you are
on, a modify-flag and the name of the text you are editing. The third thing
is the scroller\index{scroller} to the left. See also the picture below.

\begin{center}
\unitlength5mm
\begin{picture}(30,22)(0,0)
\thicklines
% Rahmen
\put(0,0){\framebox(20,13){}}

% Close Gadget
\put(1,12){\line(0,1){1}}
\put(0.4,12.4){\framebox(.2,.2){}}

% Title
\put(1.5,12.3){\tt 97/980  17 2a * XDME.tex}

% Scroller/Size
\put(0,12){\line(1,0){20}}

% Zoom | Depth
\put(18.5,12){\line(0,1){1}}

% | Zoom
\put(17,12){\line(0,1){1}}

% ____Title____
\put(19,0){\line(0,1){12}}

% Size
\put(19,1){\line(1,0){1}}
\put(19.1,0.1){\line(1,1){0.8}}
\put(19.1,0.1){\line(1,0){0.8}}
\put(19.9,0.1){\line(0,1){0.8}}

% Scroll-container and button
\put(19.2,1.2){\framebox(0.6,10.6){}}
\put(19.3,9){\framebox(0.4,2.5){}}

% Zoom
\put(17.2,12.2){\framebox(1.1,0.6){}}
\put(17.2,12.5){\framebox(0.3,0.3){}}

% Depth
\put(18.9,12.2){\framebox(0.9,0.4){}}
\put(18.7,12.8){\line(1,0){0.9}}
\put(18.7,12.8){\line(0,-1){0.4}}
\put(19.6,12.8){\line(0,-1){0.18}}
\put(18.7,12.4){\line(1,0){0.2}}

% Explanation
\thinlines
\put(1.9,19.3){\vector(0,-1){6}}
\put(1.9,19.3){\line(1,0){7}}
\put(9  ,18.8){\makebox(0,1)[l]{current line}}

\put(3.1,18.3){\vector(0,-1){5}}
\put(3.1,18.3){\line(1,0){5.8}}
\put(9  ,17.8){\makebox(0,1)[l]{lines of text}}

\put(4.6,17.3){\vector(0,-1){4}}
\put(4.6,17.3){\line(1,0){4.3}}
\put(9  ,16.8){\makebox(0,1)[l]{current column}}

\put(5.8,16.3){\vector(0,-1){3}}
\put(5.8,16.3){\line(1,0){3.1}}
\put(9  ,15.8){\makebox(0,1)[l]{char under cursor as HEX}}

\put(6.9,15.3){\vector(0,-1){2}}
\put(6.9,15.3){\line(1,0){2}}
\put(9  ,14.8){\makebox(0,1)[l]{modified flag}}

\put(8.5,14.2){\vector(0,-1){1}}
\put(8.5,14.2){\line(1,0){.4}}
\put(9  ,13.8){\makebox(0,1)[l]{title of text}}

\put(0,0){\makebox(19,12){text-area}}

\put(.5,10.8){\vector(0,1){1}}
\put(.5,10.8){\line(1,0){1.4}}
\put(2,10.3){\makebox(0,1)[l]{Close-gadget}}

\put(17.75,10.8){\vector(0,1){1}}
\put(17.75,10.8){\line(-1,0){1.65}}
\put(16,10.3){\makebox(0,1)[r]{Zoom-gadget}}

\put(17.8,9.8){\vector(1,2){1}}
\put(17.8,9.8){\line(-1,0){1.7}}
\put(16,9.3){\makebox(0,1)[r]{Depth-gadget}}

\put(16.1,8.8){\vector(1,0){2.7}}
\put(16,8.3){\makebox(0,1)[r]{Scroller}}

\put(16.1,0.5){\vector(1,0){2.7}}
\put(16,0){\makebox(0,1)[r]{Size-gadget}}

\end{picture}
\end{center}

\section{Features}\index{Features}

XDME has many features that make it one of the best editors ever (``What ? ...
No, I never had any problems with self-esteem'') including:

\begin{itemize}
\item powerful programming language with everything you need
\item full {\tt AREXX} interface
\item ability to map all keys including mouse-buttons
\item programmable menus
\item fast visual response
\item multiple windows
\item works with OS 2.0 and higher
\end{itemize}
\ \protect\nopagebreak[4]
\chapter{Invocation}\index{invocation of XDME}\index{arguments or XDME}\index{parameters of XDME}

XDME can of course be started via CLI and WB. It allows an arbitrary number
of arguments. I don't use the OS20-call for argument-parsing since
the names of the arguments are likely to interfere with names of texts.

\section{CLI-startup}\index{CLI-startup}\index{calling from CLI}\index{commandline arguments}

CLI-Usage: {\tt XDME FILES/M,SFILE=-S/K,NEW=-N/S,ICONIFY=-I/S,AUTOLOAD=-A/S
	PUBLICSCREEN=-PS/K,PROJECTFILE=-PF/K}

XDME doesn't bother for the position of the options. They can come anywhere
in the command-line.

\begin{mylist}{PUBLICSCREEN name}
\item[{\tt SFILE file}] (Startup-file): XDME reads two files on startup: S:.edrc and
	    .edrc in the local directory. With this option, you can replace
	    the local .edrc-file with some other file. The file will be read
	    automatically if it exists.
\item[{\tt NEW}] If there is already a XDME running, XDME send this process a
	    command to open a new window and load the file. If you need a new
	    process (with own AREXX-port and environment), you have to specify
	    this option.
\item[{\tt ICONIFY}] Makes XDME iconify all windows after loading the files.
\item[{\tt PUBLICSCREEN name}] XDME can open his windows on another screen.
	    You can specify the name of this screen through this option.
\item[{PROJECTFILE file}] This sets the name for the projectfile. The default
	    is {\tt XDME\_Project\_File}. If the file exists, it is ignored
	    unless you have also specified {\tt AUTOLOAD} (see below). Only
	    when you try to load a file with that name, it is treated as a
	    projectfile.

	    NOTE: You cannot edit a projectfile.
\item[{AUTOLOAD}] This turns the automatic loading of project-files on. XDME
	    will now try to load a file with the actual projectfile-name
	    from the actual directory on startup.
\end{mylist}

Anything that is not an option is treated as a filename.

\section{WB-Startup}\index{WB-startup}\index{calling from WB}\index{arguments for WB-startup}

XDME can be started via WB like any other program. Any other selected files
are read in and displayed in their own windows. You can put any of the
above options in the ToolTypes of the XDME-icon and the ToolTypes of the
selected files.

\chapter{Usage}

Any part of XDME is almost fully user-configurable. XDME consists of the
following elements:

\begin{itemize}
\item Programming Language
\item Variables in XDME
\item Menus
\item Keyboard including Mouse
\end{itemize}

\section{Programming Language}\index{programming language}

XDME has about 200 commands already build in and we'll discuss them here
and the programming language XDME understands. We made it to be simple
and fast. The basic XDME programming language is a thing between
AREXX\index{AREXX}, and C\index{C}. You can use any AREXX-command, the
style is like C and it accepts all C-expressions. DME had this, too, but it
was by far not that powerful. It didn't even came close to what XDME offers
you.

The syntax is a follows:

\begin{itemize}
\item All commands are separated with \verb/;/
\item XDME uses {\tt ()} as parentheses.
\item Strings are enclosed in \verb/""/.
\item Commands can be made joined with \verb/{}/ to a block. You can
      use a block instead of one single command.
\item Anything XDME doesn't understand is passed to AREXX.
\end{itemize}

XDME offers the following structures for flow-control:

\begin{itemize}
\item {\tt BREAK}
\item {\tt CONTINUE}
\item {\tt DO command WHILE expr}
\item {\tt FOR ( expr ; expr ; expr ) command }
\item {\tt IF expr command [ELSE command]}
\item {\tt WHILE expr command}
\item {\tt REPEAT expr command}
\item {\tt SWITCH expr \{ case\_block \}}
\end{itemize}

but no {\tt GOTO} !

\subsection{\tt BREAK}

The {\tt BREAK}-command is only useful if conjunction with a loop or
{\tt SWITCH}-statement. The purpose of the {\tt BREAK} is to break out
of some kind of loop.

\subsubsection{EXAMPLE}

\begin{verbatim}
FOR ( ; Key = GetKey () ; )
{
    IF Key == ESC THEN
	BREAK;

    ...
}
\end{verbatim}

The {\tt FOR}-loop waits for some key. If that key is {\tt ESC}, the
{\tt IF} aborts the loop. See also the example for {\tt SWITCH}.

\subsection{\tt CONTINUE}

The {\tt CONTINUE}-command is only useful if conjunction with a loop or
{\tt SWITCH}-statement. The purpose of the {\tt CONTINUE} is to stop the
execution of the body and to go to the end of the loop. That means, that
the {\tt after\_expr}essions of the {\tt FOR}- and the condition of the
{\tt DO \dots WHILE}-command get evaluated before the loop begins it's
next round. For all loops, after the {\tt CONTINUE} the condition of the
loop gets evaluated.

\subsubsection{EXAMPLE}

\begin{verbatim}
FOR ( ; Key = GetKey () ; Count++ )
{
    some_code;

    IF Key == ESC THEN
	CONTINUE;

    some_other_code;
}
\end{verbatim}

The {\tt FOR}-loop waits for some key. The {\tt some\_code} is always
executed. If the key is {\tt ESC}, the
{\tt IF} prevents the {\tt some\_other\_code} from being executed.
Instead, the execution goes on with {\tt Count++} and then with the
loop-condition.

\subsection{\tt DO \dots WHILE}

The {\tt DO \dots WHILE}-command is like the {\tt WHILE}-command, but the
expression is evaluated at the end of the loop thus the loop is
executed at least once.

\subsubsection{EXAMPLE}

\begin{verbatim}
DO
    left;
WHILE (column > 5)
\end{verbatim}

Moves the cursor to the left and checks if it is still to the right of the
5th column. Note that the cursor is also moved, if it's already to the left
of the 5th column !

\subsection{\tt FOR}

The {\tt FOR}-command is the most powerful loop-command XDME offers.
It has several arguments. For more clarity look at this:

\centerline{\tt FOR ( before\_expr ; loop\_expr ; after\_expr ) command }

As you can see, the three expressions have different meanings. The
{\tt before\_expr} is evaluated before the loop actually starts.
It is usually used for initializing counters. The second one,
the {\tt loop\_expr}, determines the condition that ends the loop. If this
expression gets zero, the loop terminates. The last expression, the
{\tt after\_expr}, is called every time the whole body of the
loop has been processed. It is usually used for incrementing counters
and the like.

This type of loop behaves exactly like the following lines:

\begin{verbatim}
before_expr;

WHILE loop_expr
{
    command;

    after_expr;
}
\end{verbatim}

\subsubsection{EXAMPLE}

\begin{verbatim}
FOR ( counter=0 ; counter<1000 ; counter++ )
    title "Counter " counter;
\end{verbatim}

Sets {\tt counter} to zero and increments it up to 999. The progress is
shown in the title-bar.

\subsection{\tt IF}

The {\tt IF}-command takes two or four parameters. The first parameter is
an expression that is evaluated. If the expression evaluates to
something not zero, the second parameter is executed. Otherwise
{\tt IF} looks for an {\tt ELSE} and if it finds one, the part after
the {\tt ELSE} is executed.

\subsubsection{EXAMPLE}

\begin{verbatim}
IF (column > 5)
    title "Column > 5";
ELSE
    title "Column <= 5";
\end{verbatim}

If the cursor is to the right of column 5, the title is set to
``X $>$ 5''. Else the title is set to ``X $<=$ 5''.

\subsection{\tt WHILE}

The {\tt WHILE}-command has two parameters. While the {\tt expr} is
non-zero, the second parameter {\tt command} gets executed.

The difference to the {\tt REPEAT}-command is that {\tt WHILE}
re-calculates the value of {\tt expr} before every loop.
If the expression evaluates to zero, {\tt WHILE} stops. That also
means that, if the expression is already or always zero, the {\tt WHILE}-body
gets never executed.

\subsubsection{EXAMPLE}

\begin{verbatim}
WHILE (column > 5)
    left;
\end{verbatim}

As long as the cursor is to the right of the 5th column, it is moved to
the left, resulting in a cursor in column 5. If the cursor is already
in the 5th column or to the left, nothing happens.

\subsection{\tt REPEAT}

The {\tt REPEAT}-command has two parameters. The value of {\tt expr} is
evaluated and the second parameter {\tt command} gets executed {\tt expr}
times.

The difference to the {\tt WHILE}-command is that {\tt REPEAT}
calculates the value of {\tt expr} only once and takes the result as
``the number of loops''.

If {\tt expr} is zero or less, the {\tt REPEAT} does nothing.

\subsubsection{EXAMPLE}

\begin{verbatim}
REPEAT (x - 5)
    left;
\end{verbatim}

The distance of the cursor to the left column is calculated and then
the cursor is moved the appropriate times to the left.

\subsection{\tt SWITCH}

The {\tt SWITCH}-command is one of the most complex control-commands XDME
has to offer. It has two arguments, an {\tt expr}ession and a block
of {\tt CASE}-statements. The {\tt expr}ession is evaluated and the
result is compared with every {\tt CASE-expr}ession. If we encounter
a {\tt CASE-expr}ession that matches the {\tt SWITCH-expr}ession,
ie. when both are numerically equal, the body of the {\tt CASE}-statement is
executed.

Since every {\tt CASE}-statement is only a label, execution is continued
from there and up to the end of the {\tt CASE}-block. If you want to
stop the execution and get out of the {\tt SWITCH}, you have to use
{\tt BREAK}. Also you can use {\tt CONTINUE} to re-evaluate the
{\tt SWITCH-expr}ession and start over.

The last thing that is important about the {\tt SWITCH} is the
{\tt DEFAULT}-statement. If none of the {\tt CASE}-statements match and
there is a {\tt DEFAULT}-statement, the {\tt DEFAULT}-body gets executed.

Phew, this heavy stuff, so we'll try an example:

\subsubsection{EXAMPLE}

\begin{verbatim}
SWITCH (Key = GetKey ())    /* Wait for a key to be pressed and
			       remember that key */
{
    CASE Return:	    /* If user pressed RETURN */
	title "RETURN was pressed";  /* tell that */
    BREAK;		    /* and break out of switch */

    CASE F1..F10:	    /* F-Keys are not allowed ! */
	error "Huh ? You cannot use the function-keys now !"
    CONTINUE;		    /* Start over */

    CASE '4':               /* Count 4 times */
	Count ++;

    CASE '3':               /* Count 3 times */
	Count ++;

    CASE '2':               /* Count 2 times */
	Count ++;

    CASE '1':               /* Count 1 time */
	Count ++;
    BREAK;		    /* stop it */

    DEFAULT:
	error "You pressed an unsupported key !"
    CONTINUE;
}
\end{verbatim}

Well, most things are in the commands but a few. The second {\tt CASE}
shows the usage of {\tt RANGE}s. The four {\tt CASE}-statements that
accept numbers are not terminated thus if you press {\tt 4}, all four
{\tt Count ++}-statements are processed. The last thing is that, if you
press a key that is not recognized by any case, the {\tt DEFAULT}
tells you just that and repeats the {\tt SWITCH} one more time.

\section{Variables in XDME}

XDME offers a variety of variables which can be internal and external.
Variables can be stacked and you can calculate with them:

\begin{itemize}
\item Declaring variables
\item Scope of variables in XDME
\item Expressions in XDME
\end{itemize}

You should also have a look at the chapter ``Macros in XDME'' on page
\pageref{Macros in XDME}.

\subsection{Declaring variables}

XDME knows the following types for variables:

\begin{center}
\begin{tabular}{|l|l|} \hline
 {\tt CHAR}	& ASCII-characters			\\
 {\tt INTEGER}	& integer class ($-2^{31}..2^{31}-1$)   \\
 {\tt STRING}	& a string of text			\\
 {\tt FLOAT}	& a floating-point variable		\\
% {\tt KEY}	 & a key				 \\
\hline
\end{tabular}
\end{center}

%
%and the following compound types:
%
%\begin{center}
%\begin{tabular}{|l|l|}
%\hline
% Array        & a number of elements of the same type \\
% {\tt ENUM}   & an enumeration \\
% {\tt LIST}   & a list of elements of the same type \\
% Pointers     & a pointer to an element \\
% {\tt STRUCT} & multiple different elements together \\
% {\tt UNION}  & like {\tt STRUCT}, but can contain only one \\
%	       & element at any time \\
%\hline
%\end{tabular}
%\end{center}
%
%The last thing are storage-qualifiers. XDME offers those to you:
%
%\begin{center}
%\begin{tabular}{|l|l|}
%\hline
% {\tt CONST}  & for constant variables \\
% {\tt EXTERN} & for external/global variables \\
% {\tt STATIC} & for static variables \\
%\hline
%\end{tabular}
%\end{center}
%
%Now this seems a bit difficult, but XDME's concept of variables is quite
%straight-forward.
%
%\subsection{Integral types}
%
%In a {\tt CHAR} you can store one character of text. A {\tt STRING} is
%an array of characters, but a {\tt STRING} in XDME can be of
%arbitrary length. The {\tt INTEGER} type can hold any integral value
%while {\tt FLOAT} is for all numbers with decimal point. The {\tt KEY}
%can hold one key.
%
%\subsection{Compound types}
%
%The compound types help in building up more complex structures.
%The type {\em array} is represented with a pair of brackets after the
%variable-name like {\tt INTEGER a}$[10]$. Elements in an array can be
%accessed by their number. The counting starts with 0. Unlike in other
%programming languages, negative numbers are valid ! They specify an offset
%from the end of the list, ie. {\tt a}$[-1]$ is the same as {\tt a}$[9]$
%in the above example. Unlike C, XDME will prevent you from accessing
%array-elements, that are out of range, ie. {\tt a}$[10]$ will result
%in an error.
%
%The {\tt ENUM}eration is a list of names that create a type. This
%type can contain only those values that are present within the {\tt ENUM}.
%A famous example are colors:
%
%\begin{verbatim}
%ENUM colors
%{
%    BLACK,
%    RED,
%    BLUE,
%    GREEN,
%    WHITE
%};
%\end{verbatim}
%
%This creates a new type called {\tt colors} that can be set to {\tt BLACK},
%{\tt RED}, {\tt BLUE}, {\tt GREEN} and {\tt WHITE}. You can now declare
%a variable of type {\tt colors}: {\tt colors x;} and can assign
%values to {\tt x}. {\tt x = 1;} will yield in an error.
%
%{\tt LIST}s are made of one type but, unlike arrays, have an arbitrary
%number of elements. You can add and remove elements from a list at any
%time; the list grows and shrinks accordingly. Like arrays, elements in the
%list can be accessed by their number or by an internal pointer.
%This pointer can be set by the commands {\tt FIRST}, {\tt NEXT},
%{\tt PREVIOUS}, {\tt LAST} and some more. The convention for numerical
%access is like in an array, ie. the element -1 is the last in the list.
%
%Pointers are a very powerful tool for both programming and bug-making.
%If you are not familiar with C-pointers, you should not use them.
%A pointer is declared by using a star like in {\tt INTEGER * a;}.
%This makes a pointer called {\tt a} that points to an {\tt INTEGER}
%{\bf or} it can be interpreted as the base-address of an array of
%integers. Unlike in C, XDME will check if you try to use an uninitialized
%pointer.
%
%{\tt STRUCT}s and {\tt UNION}s are very similar to each other. Both
%are made up from several different elements. The difference is that
%a {\tt STRUCT}ure can hold several different elements at any time, while
%a {\tt UNION} can only hold one of those elements at any time.
%

\subsubsection{Accessing ENV:-Variables}

All ENV:-variables can be accessed from within XDME without any special
notation. If you have, for example, an ENV:-variable called {\tt DATE},
you can view it with {\tt TITLE DATE}. All ENV:-variables are treated as
strings. If you assign such a variable to an internal XDME-variable,
XDME automatically tries to convert the string to the type that's needed
for the internal variable. The same applies to writing some value in
ENV:-variables.

\subsection{Scope of variables in XDME}

A variable in XDME is known only inside the actual block and in all blocks
that are in this block (like in C). If you declare a variable outside
a function, it's a global variable which is recognised everywhere since
every block is in that level.

\subsection{Expressions in XDME}

XDME accepts every C-construct. Unlike C, XDME does an internal
type-conversion. It is perfectly possible to add a number to
a string or a string to a number. In the first case, the number
is converted to a string and appended to the string and in the latter,
the string is converted to a number (ie. we look if the string contains
a number and make this number to an int or float) and numerically added
to the number.

\section{Menus}

\subsection*{Menus Overview}
    XDME allows to add (almost) any number of menus with infinite number of
    entries. The following commands are available for manipulating menus:

    \begin{itemize}
	\item {\tt MENUADD	 menuname itemname	   command}
	\item {\tt MENUCHECK	 menuname itemname	   command}
	\item {\tt MENUBAR	 menuname}
	\item {\tt MENUDEL	 menuname itemname}
	\item {\tt MENUDELBAR	 menuname}

	\item {\tt MENUSUBADD	 menuname itemname subname command}
	\item {\tt MENUSUBCHECK  menuname itemname subname command}
	\item {\tt MENUSUBBAR	 menuname itemname}
	\item {\tt MENUDELSUB	 menuname itemname subname}
	\item {\tt MENUDELSUBBAR menuname itemname}

	\item {\tt MENUSETITEM menuname itemname	 status}
	\item {\tt MENUCHKITEM menuname itemname	 variablename}
	\item {\tt MENUSETSUB  menuname itemname subname status}
	\item {\tt MENUCHKSUB  menuname itemname subname variablename}

	\item {\tt MENUDELHDR	 menuname}
	\item {\tt MENUCLEAR}

	\item {\tt MENUOFF}
	\item {\tt MENUON}

	\item {\tt MENUSAVE filename}
	\item {\tt MENULOAD filename}

	\item {\tt NEWMENUSTRIP name}
	\item {\tt USEMENUSTRIP name}
	\item {\tt REMMENUSTRIP}
    \end{itemize}

\subsection{\tt MENUADD}

	This command takes three arguments. The name of the menu which you want to
	add the new item. The menu is created if it does not already exist.
	The next parameter if the name of the item that is to be added to the
	menu and the last parameter is the command that is to be executed if
	the user selects the item. For now, there are no subitems.

	If you have a key that does the same as the menu, it's recommended to
	add the mapping of the key after the name of the item (for the form
	of key-mappings, see the page \pageref{mapping keys}).

    \subsubsection{EXAMPLE}

	{\tt MENUADD Edit `Read C-Config' `source DATA:c.xdme'}

	This creates the menu {\tt Edit} with an item {\tt Read C-Config} or just adds the
	item {\tt Read C-Config} to the menu {\tt Edit} if it's already there. The command
	{\tt source DATA:c.xdme} gets executed, if the user selects that item.
	(This macro reads a file and executes all commands in that file).


\subsection{\tt MENUCHECK}

	This command does almost the same as {\tt MENUADD} despite, that it creates
	not only a simple Commanditem, but a toggleitem, which can be toggled with
	the {\tt MENUSETITEM} and checked with the {\tt MENUCHKITEM}.
	there is currently no way to connect a variable immediately to a checkitem
	in order to make the checkitem represent the variable's value or vice versa.

    \subsubsection{EXAMPLE}
	Ask Aaron ...

\subsection{\tt MENUBAR}

	This command does almost the same as {\tt MENUADD} despite that it appends
	only a simple non-functional separatorbar to the specified menu; since such
	a

    \subsubsection{EXAMPLE}
	Ask Aaron ...

\subsection{\tt MENUDEL}

    This command is the opposite of {\tt MENUADD} or {\tt MENUCHECK}. It takes two arguments which
    specify an item in a menu that is to be removed. Note that the menu
    is also deleted, if that item was the last in the menu.
    {\tt MENUDEL} may also be used to delete a submenucarrier and all of its subitems.

\subsubsection{EXAMPLE}

    {\tt MENUDEL Edit `Read C-Config'}

    This removes the item we created above.


\subsection{\tt MENUDELBAR}

	This command is the opposite of {\tt MENUBAR}. It takes one argument which specifies a menu
	and deletes the first Itembar found in that menu. (see also {\tt MENUDEL})

    \subsubsection{EXAMPLE}
	Ask Aaron ...

\subsection{\tt MENUSETITEM}

	change the status of a menuitem, which was created with the command {\tt MENUCHECK}.
	the last argment - status may be one of "set" "reset" or "toggle".
	I must admit, that the current name is not the best, but I had no other ideas ...

    \subsubsection{EXAMPLE}
	Ask Aaron ...

\subsection{\tt MENUCHKITEM}

	get the current status of a menuitem which was created with the command {\tt MENUCHECK} into a variable, which is given by its name.
	I must admit, that the current name is not the best, but I had no other ideas ...
	This command is used in XDME <2.0 to check the status of a Checkitem,
	since we do not have functions in our command language at that time ...
	so we have to fetsch the status into a variable and examine the value of that variable;
	the command is obsolete under XDME >= 2.0, or as soon, as the C-Interpreter
	is ready to use.
	Note, that usage of that command to an item, which was not created with {\tt MENUCHECK} is undefined.

    \subsubsection{EXAMPLE}
	Ask Aaron ...

\subsection{\tt MENUSUBADD}

	This command takes four arguments. The name of the menu and the item which you want to
	add the new subitem. The menu and the menuitem are created if they do not already exist.
	The next parameter is the name of the subitem that is to be added to the
	menu and the last parameter is the command that is to be executed if
	the user selects the item. subitems cannot be added to menuitems which were created with
	{\tt MENUADD} or {\tt MENUCHECK}.

	If you have a key that does the same as the menu, it's recommended to
	add the mapping of the key after the name of the item (for the form
	of key-mappings, see the page \pageref{mapping keys}).

    \subsubsection{EXAMPLE}
	Ask Aaron ...

\subsection{\tt MENUSUBCHECK}

	This command does almost the same as {\tt MENUSUBADD} despite, that it creates
	not only a simple Commanditem, but a Toggleitem, which can be toggled with
	the {\tt MENUSETSUB} and checked with the {\tt MENUCHKSUB}.
	there is currently no way to connect a variable immediately to a checkitem
	in order to make the checkitem represent the variable's value or vice versa.

    \subsubsection{EXAMPLE}
	Ask Aaron ...

\subsection{\tt MENUSUBBAR}

	This command does almost the same as {\tt MENUSUBADD} despite that
	only a simple non-functional separatorbar to the specified submenu; since such
	a bar has no name and no connected command, it has two arguments less.

    \subsubsection{EXAMPLE}
	Ask Aaron ...

\subsection{\tt MENUDELSUB}

	This command is the opposite of {\tt MENUSUBADD} or {\tt MENUSUBCHECK}. It takes three arguments which
	specify an subitem in a menu that is to be removed. Note that the menu or the item
	is also deleted, if that subitem was the last in the menuitem and so on.

    \subsubsection{EXAMPLE}
	Ask Aaron ...

\subsection{\tt MENUDELSUBBAR}

	This command is the oppos
	which must be a subitemcarrier and deletes the first Subitembar found in that menuitem. (see also {\tt MENUDELSUB})

    \subsubsection{EXAMPLE}
	Ask Aaron ...

\subsection{\tt MENUSETSUB}

	change the status of a submenuitem, which was created with the command {\tt MENUSUBCHECK}.
	the last argment - status may be one of "set" "reset" or "toggle".
	I must admit, that the current name is not the best,

    \subsubsection{EXAMPLE}
	Ask Aaron ...

\subsection{\tt MENUCHKSUB}

	get the current status of a submenuitem, which was created with the command {\tt MENUSUBCHECK} into a variable, which is given by its name.
	I must admit, that the cu
	This command is used in XDME <2.0 to check the status of a Checkitem,
	since we do not have functions in our command language at that time ...
	so we have to fetsch the status into a variable and examine the value of that variable;
	the command is obsolete under XDME >= 2.0, or as soon, as the C-Interpreter
	is ready to use.
	Note, that
	there is currently no way to connect a variable immediately to a checkitem
	in order to make the checkitem represent the variable's value or vice versa.

    \subsubsection{EXAMPLE}
	Ask Aaron ...


\subsection{\tt MENUSAVE}
	Save the complete current menustrip to a specified file;
	this operation might use a special file format, though we are currently
	using Asciifiles, but this might change ...
	saved menues can easily be read with the next command

    \subsubsection{EXAMPLE}
	Ask Aaron ...

\subsection{\tt MENULOAD}
	read in a complete menustrip from a specified file; the current menustrip
	is cleared before that action.

    \subsubsection{EXAMPLE}
	Ask Aaron ...

\subsection{\tt NEWMENUSTRIP}
	Create a new menustrip with the given name;
	XDME's menues are organised as menustrips, so that You can use different
	menustrips for different windows; the default menustrip that is created
	on startup is called ``default''.
	Normal user will perhaps never use that feature (it had been
	implemented for over a year before someone decided to document it).

    \subsubsection{EXAMPLE}
	Ask Aaron ...

\subsection{\tt USEMENUSTRIP}
	Activate another menustrip for the current window (and for all those
	windows, which currently do not have a fixed menustrip).

    \subsubsection{EXAMPLE}
	Ask Aaron ...

\subsection{\tt REMMENUSTRIP}
	Delete the current menustrip

    \subsubsection{EXAMPLE}
	Ask Aaron ...



\subsection{\tt MENUDELHDR}

	If you want to remove the whole menu, you can try several {\tt MENUDEL}-commands
	of use this one. It removes the whole menu at once and takes only one
	parameter which is the name of the menu that is to be removed.

    \subsubsection{EXAMPLE}

	{\tt MENUDELHDR Edit}

	Makes the whole Edit-menu disappear.


\subsection{\tt MENUCLEAR}

	Sometime you might need to remove ALL menus at once. Guess it, this is the
	command at hand. It removes all menus.

    \subsubsection{EXAMPLE}

	What ???? An example for that easy command ??? Are you NUTS ?? No
	imagination anymore ?

\subsection{\tt MENUOFF}

	This is the opposite of {\tt MENUON}. {\tt MENUOFF} switches the menus off temporarily.
	That means that the menus are stored internally, but not accessible by the
	user anymore. Note that any mapping for the right mousebutton now becomes
	active ! This is normally {\tt iconify window}. Use this command, if you
	make changes to the menus thus the user cannot interrupt the creation
	of the new entries by selecting from the menu.

	Note that multiple calls are stacked, ie. if you call {\tt MENUOFF} five times, you
	need five calls to {\tt MENUON} to enable the menus again.

    \subsubsection{EXAMPLE}

	Obvious, no ?

\subsection{\tt MENUON}

	This is the opposite of {\tt MENUOFF}. {\tt MENUON} switches the menus on again.
	That means that the menus made accessible by the
	user again. Note that any mapping for the right mousebutton now becomes
	inactive ! Use this command, if you
	made changes to the menus and called {\tt MENUOFF} to do so undisturbed.

	Note that multiple calls are stacked, ie. if you call {\tt MENUOFF} five times, you
	need five calls to {\tt MENUON} to enable the menus again.

    \subsubsection{EXAMPLE}

	Try it yourself :-)






\section{Keyboard including Mouse} \label{mapping keys}

\subsection*{Keyboard Overview}

    We now enter the most {\em simple} part of XDME ... This is of course the most
    powerful part :-): The mapping of keys and buttons. As mentioned above,
    XDME has the ability to map functions to all keys. These {\em functions}
    can be simple text that is to be inserted up to sophisticated programs
    that aid you in your work. We only have a few commands here:

    \begin{itemize}
	\item {\tt UNMAP key}
	\item {\tt MAP key command [comment]}
	\item {\tt KEYSAVE filename}
	\item {\tt KEYLOAD filename}
	\item {\tt NEWKEYTABLE name}
	\item {\tt USEKEYTABLE name}
	\item {\tt REMKEYTABLE}
	\item {\tt QUALIFIER qualifers}
    \end{itemize}

    but at least the first 2 of them are really complicated to use. Don't worry though. I'll try to
    guide you through this safely :-)

    first we will check the simple commands ...

\subsection{\tt KEYSAVE}
	Save the complete current Keytable to a specified file;
	though that file is currently in Ascci, we might decide
	to change that to a special file format in order to speed
	things up.

    \subsubsection{EXAMPLE}
	Ask Aaron ...

\subsection{\tt KEYLOAD}
	Read a Keytable saved with the {\tt KEYSAVE} command, from a specified file.
	Previously we delete the active Keytable, so a corrupted file might lead You
	into trouble ...

    \subsubsection{EXAMPLE}
	Ask Aaron ...

\subsection{\tt NEWKEYTABLE}
	Create a new Keytable of a given name;
	the dafault keytable created on startup is called ``default''.
	Like Menues, we can use different Keytables for different windows,
	but normal user will perhaps never use that feature (it had been
	implemented for over a year before someone decided to document it).

    \subsubsection{EXAMPLE}
	Ask Aaron ...

\subsection{\tt USEKEYTABLE}
	Use the Keytable of the specified name (if exists)

    \subsubsection{EXAMPLE}
	Ask Aaron ...

\subsection{\tt REMKEYTABLE}
	Delete the current Keytable. (no arguments)

    \subsubsection{EXAMPLE}
	Ask Aaron ...


\subsection{\tt MAP}

	Maps a command to a key. Simple, eh ? Oh well, not quite :-) You can specify
	the key by a list of qualifiers that have to be pressed (like {\tt Shift},
	{\tt Ctrl}, {\tt Alt}) and the key itself.

	The next parameter is the command. If you want to map more than one command,
	you have to use {\tt\{} and {\tt\}} to form a block.

	The last parameter is optional and contains a short description of the
	mappings that explains what it does. XDME expects a string here.

	The parameter {\em key} is a bit complicated but the definition is
	quite straight-forward. The sections following the command descriptions
	describe the possible forms of a {\em key} parameter.

\subsection{\tt UNMAP}

	This command undoes the mapping of a key. It takes one parameter, the
	key you want to reset. This is exactly the same as in {\tt MAP} (see below).
	{\tt UNMAP}ing a key means not to make it a ``do-nothing'', but
	resetting it to it's default state. If you, for example, mapped
	the key {\tt M} with some garbage and want to undo it, just
	enter the command-line-mode and type {\tt unmap m} or better
	you would try to do that and realize that you can neither type
	{\tt unmap} nor {\tt m} because you have no {\tt m}. Well,
	good luck then ! (Just a joke, don't worry, in command-line-mode
	the normal alpha-chars are remapped to their default meanings)

\subsection{\tt QUALIFIER}

	XDME has the ability to simulate multi-keystroke sequences;
	in order to achieve that goal, You can specify a key-sequence,
	that activates a special qualifier, and another one, that uses
	that qualifier...
	that feature was added to enable EM*cs like Control-X sequences

    \subsubsection{EXAMPLE}

	MAP c-x (QUALIFIER x) \\
	MAP x-s (SAVEOLD) \\
	Now You can save the current file with the following actions:
	press ``Control'', keep it pressed, press ``X'', release ``X'',
	release ``Control'', finally press ``S''.

\subsection{Qualifiers}

    A list of available Qualifiers and their meanings can be found on page
    \pageref{list of qualifiers}. Note that qualifiers, unlike keys, are
    case-sensitive.

    \begin{table} \centerline{\bf List of all qualifiers that XDME knows of}
    \begin{center}
    \begin{tabular}{|l|l|} \hline
	  {\tt s} & {\tt Shift} and {\tt CapsLock}    \\
	  {\tt c} & {\tt Ctrl}			 \\
	  {\tt a} & {\tt Alt}			 \\
	  {\tt A} & {\tt Amiga} 		 \\
	  {\tt L} & {\tt Left Mousebutton}	 \\
	  {\tt M} & {\tt Middle Mousebutton}	 \\
	  {\tt R} & {\tt Right Mousebutton}	 \\
	  {\tt x}, {\tt y}, {\tt z} & special qualifiers only settable via {\tt QUALIFIER}\\
	  {\tt u} & the release of the key \\ \hline
    \end{tabular}
    \end{center}
    \label{list of qualifiers}
    \end{table}

    EXCEPTION: The {\tt s} qualifier can also be specified for letters by
    using the capital representation ! Therefore {\tt s-a} and {\tt A} are the
    same mapping (See below) !

    Notes:
    \begin{itemize}
    \item    The {\tt CapsLock}-key is only for alpha-keys (like {\tt a}, {\tt b}, ...)
	the same as shift. It doesn't work for numbers for example.

    \item    You cannot use the right mousebutton if you use menus. In this
	case, Intuition removes all events concerning the right mousebutton
	and displays the menus instead ! Also note that
	XDME has three special qualifiers that are not keys but can be set via
	a command ! This allows you to map the same keys according to your needs
	without the need to remap them if you need another function. Like when you
	have a text-editing and a program-editing mode with different functions
	for the {\tt F1}-key for example.
    \end{itemize}

    All qualifier-keys must be pressed for the test to succeed. {\tt asc} means
    ``any {\tt Shift} or {\tt CapsLock} with {\tt Ctrl} and any {\tt Alt}'',
    ie. XDME does not
    differentiate between the left and the right-keys. Thus, {\tt ascALMR} is a
    possible qualifier and you can use that, but you should be aware, that
    the user might get into difficulties if you also specify a key in the right
    part of the keyboard that is to be pressed like {\tt HELP} :-)

    The {\tt u} qualifier says that you don't want the ``key press'' but the
    ``key release'' event. This allows two different action on press and on
    release. You can for example start to mark a block with the press of the
    key and end the selection of the key-release.

    Now we have the qualifiers and come to the specification of the key itself.
    The ``key'' is in most cases the letter that is on the key-cap or the string
    of letters on the cap. (eg. {\tt b} is the {\tt B}-key, {\tt F1} is the
    {\tt F1}-key, {\tt B} is the {\tt B}-key with {\tt Shift} !).

\subsection{Keys}

    On the page \pageref{list of keys} is a list of all available keys.
    All keys not mentioned there can be specified by simply pressing them.

    \begin{table}\centerline{\bf List of all keys that XDME knows of}
    \begin{center}
    \begin{tabular}{|l|l|} \hline
	{\tt a}-{\tt z}, {\tt 1}-{\tt 9}, ...	&  Just these keys				   \\
	{\tt A}-{\tt Z} 	    & letter-keys with {\tt Shift}			     \\
	{\tt F1}-{\tt F10}	    &  Function-Keys				       \\
	{\tt ESC}	      & the {\tt ESC}-key				    \\
	{\tt TAB}	      & the {\tt TAB}-key (below {\tt ESC})                          \\
	{\tt BS}	      & the backspace key (above {\tt Return})                \\
	{\tt RETURN}	      & the largest key on the keyboard 		 \\
	{\tt DEL}	      & the {\tt Delete} key				       \\
	{\tt HELP}	      & the {\tt Help} key				       \\
	{\tt UP},{\tt DOWN},{\tt LEFT},{\tt RIGHT} & the cursor-keys				   \\
	{\tt NK0}-{\tt NK9}	    & the numbers on the numeric keypad 	       \\
	{\tt NK.}	      & the {\tt .}-key on the numeric pad			 \\
	{\tt ENTER}	      & the Enter-key on the numeric pad		 \\
	{\tt NK-},{\tt NK+},{\tt NK*},{\tt NK/} & the {\tt -}, {\tt +}, {\tt *} and {\tt /}-keys on the numeric pad	\\
	{\tt NK[},{\tt NK]}	    & the {\tt [} and {\tt ]} on the German numeric pad        \\
	{\tt NK(},{\tt NK)}         & the same keys on the English/USA and         \\
	{\tt NK@},{\tt $\makebox{NK}^\circ$}	  & on the Canadian numeric pad 		 \\
	{\tt LMB}, {\tt MMB}, {\tt RMB}   & the left, middle and right mousebutton	     \\
	{\tt MMOVE}	      & the mouse-movement				 \\ \hline
    \end{tabular}
    \end{center}
    \label{list of keys}
    \end{table}

    Notes:
    \begin{itemize}
    \item Although all keys are listed uppercase here, XDME accepts them
	in both upper- and lowercase. Also mixed case is allowed, too :-)
    \item The mappings for the first two keys in the upper left edge of the
	numeric pad might be wrong. If this is true for you, tell us your
	mappings and you keyboard/country and we'll fix it.
    \item There is no way to map ``dead-keys''.
    \end{itemize}

    A full key-specification consists of two parts: the qualifiers and the
    key itself.

\subsection{EXAMPLES}

    Example key-mappings can be found on page \pageref{example mappings}.

    \begin{table}\centerline{\bf Examples for key-mappings}
    \begin{center}
    \begin{tabular}{|l|l|} \hline
	{\tt bs} or {\tt BS}	& only the backspace-key			\\
	{\tt b} 		& the key with the {\tt B} on the cap		\\
	{\tt B} or {\tt s-b}	& the key {\tt B} with {\tt Shift}		\\
	{\tt f1} or {\tt F1}	& the {\tt F1}-key				\\
	{\tt a-a}		& {\tt Alt A}					\\
	{\tt A-a}		& {\tt Amiga A} 				\\
	{\tt sA-a} or {\tt A-A} & {\tt Shift Amiga A}				\\
	{\tt L-lmb}		& the left mousebutton pressed			\\
	{\tt uL-lmb}		& the left mousebutton released 		\\
	{\tt L-mmove}		& pressing the {\tt LMB} while moving
				  the mouse					\\
	{\tt LR-lmb}		& left mousebutton hit while right		\\
				& mousebutton is help down			\\
	{\tt s-.}		& {\tt Shift .} ({\tt >} for USA keyboards,
				  {\tt :} for Germans)			    \\
	{\tt ` '}               & the {\tt Space}-bar                           \\
	{\tt s-` '}             & {\tt Shift Space}                             \\ \hline
    \end{tabular}
    \end{center}
    \label{example mappings}
    \end{table}

    Ok. This is a bit much stuff so here some more detailed explanations:

\subsection{Mapping Mousebuttons}

    Mousebuttons serve both as QUALIFIERS and KEYS. This is a bit complicated
    and that you can also map mousemovements doesn't help a bit. But it's really
    simple. Lets look at two examples

    \hspace*{1cm} {\tt L-a} and {\tt L-lmb}

    The first one maps the key ``{\tt a} while the left mousebutton is pressed''
    , the
    second one the left mousebutton itself. This is because of Intuition.
    If you press a mousebutton, Intuition always sends a message that you have
    pressed it AND sets the qualifier for ``left mousebutton''. Thus to
    ask for the any mousebutton, you have to specify both the KEY and the
    according QUALIFIER.

    Now what happens if you do a mapping on any mousebutton and also use
    this button as a qualifier in another mapping like:

    \begin{center}
    \begin{verbatim}
    map L-lmb tomouse	`move the cursor to the mouse'
    map L-a ``hello''   `insert "hello" while LMB and a'
    \end{verbatim}
    \end{center}

    The first mapping get always executed if you press the left mousebutton.
    If you press {\tt A} after that while still pressing the left mousebutton,
    the second mapping gets executed. That is, BOTH mappings get executed
    even if you only want {\tt L-a}. To avoid this, better do NOT use the mousebuttons
    as qualifiers if you are not absolutely sure what you want. There
    are enough qualifiers on the keyboard.

    The next thing is mousemovement. As mentioned before, you can also
    map the movement of the mouse:

    \begin{center}
    \begin{tabular}{ll}
	{\tt L-mmove}  & moving the mouse while left mousebutton is held down  \\
	{\tt mmove}    & just while moving the mouse. BE CAREFUL !	       \\
	{\tt LR-mmove} & both Mousebuttons pressed while moving the mouse      \\
    \end{tabular}
    \end{center}

    Note: I don't see an actual purpose for mapping the movement of the
	mouse WITHOUT any qualifiers, but it's possible to do that. But be
	careful ! XDME will execute the command every time you touch your
	mouse !

\subsection{Default Mappings}

    XDME does some default mappings for you on startup. These map all
    printable ASCII keys to just insert the ASCII-equivalent of the
    key. Also all extra keys like {\tt RETURN}, {\tt BS}, {\tt DEL}, etc. are mapped
    to their meaning. Here is a list of other mappings that are done:

    \begin{table}\centerline{\bf Default Key-Mappings of XDME}
    \begin{tabular}{|l|p{2in}|p{2.8in}|}
    \hline
	    Key & Mapping & Description \\
    \hline
	    {\tt A-n} &      {\tt newwindow arpload} & Open new Window and show file\-re\-quester \\
	    {\tt A-o} &      {\tt arpload} & Load a new file via file-requester \\
	    {\tt A-s} &      {\tt saveold} & Save file (overwrite any old file) \\
	    {\tt A-a} &      {\tt arpsave} & Save file as ... \\
	    {\tt A-q} &      {\tt quit} & Quit one XDME-Window or XDME, if this was the last window \\
	    {\tt A-Q} &      {\tt quitall} & Quit XDME. You are asked if you want to save any unsaved work \\
	    {\tt A-z} &      {\tt undo} & Undo \\
	    {\tt A-x} &      {\verb/bsave t:xdme\_clip bdelete/} & Move block to Clip \\
	    {\tt A-c} &      {\verb/bsave t:xdme\_clip unblock/} & Copy block to Clip \\
	    {\tt A-v} &      {\verb/insfile t:xdme\_clip/} & Paste block from Clip \\
	    {\tt A-f} &      {\verb/escimm (findstr \`) next/} & Ask and search for string \\
	    {\tt A-r} &      {\verb/escimm (findstr \`)/} & Ask, search and replace strings \\
		      &  {\verb/escimm (repstr \`) nextr/} & \\
	    {\tt A-e} &      {\verb/escimm (execute \`)/} & Execute DOS-Command \\
	    {\tt A-b} &      {\tt block} & Mark beginning or end of block \\
	    {\tt A-i} &      {\tt iconify} & iconify actual XDME-window \\
    \hline
	    {\tt a-up} &     {\tt scrollup} & Scroll actual window up one line \\
	    {\tt a-down} &   {\tt scrolldown} & Scroll actual window down one line \\
	    {\tt a-left} &   {\tt wleft} & Jump to beginning of next word \\
	    {\tt a-right} &  {\tt wright} & Jump to beginning of previous word \\
    \hline
	    {\tt s-up} &     {\tt pageup} & Jump to top of screen or scroll one page up \\
	    {\tt s-down} &   {\tt pagedown} & Jump to bottom of screen or scroll one page down \\
	    {\tt s-right} &  {\tt pageright} & Jump to end-of-line, right border or scroll one page right \\
	    {\tt s-left} &   {\tt pageleft} & Jump to left border, beginning-of-line or scroll one page left \\
	    {\tt s-return} & {\tt firstnb down} & Go to beginning of line and then one line down \\
	    \verb/s- / &      {\verb/( )/} & Make {\tt Shift Space} a simple {\tt Space } \\
    \hline
    \end{tabular} \\
    continued on the next page ...
    \label{default mappings}
    \end{table}

    \begin{table}\centerline{\bf Default Key-Mappings of XDME (continued)}
    \begin{tabular}{|l|p{2in}|p{2.8in}|}
    \hline
	    Key & Mapping & Description \\
    \hline
	    {\tt c-up} &     {\tt top} & Go to beginning of text \\
	    {\tt c-down} &   {\tt bottom} & Go to end of text \\
	    {\tt c-right} &  {\tt last} & Go to end-of-line \\
	    {\tt c-left} &   {\verb/set tmp $colno firstnb/}
			     {\verb/if x=$tmp first/} &
					Jump to beginning of line or to the far left \\
    \hline
	    {\tt c-esc} &    {\tt recall} & Recall last commandline \\
	    {\tt c-c} &      {\verb/""/} & On {\tt Ctrl C} do nothing (This is handled internally in XDME) \\
	    {\tt c-j} &      {\tt join} & Join two lines together \\
	    {\tt c-i} &      {\tt insertmode toggle} & Switch between {\tt INSERT}- and {\tt OVERWRITE}-mode \\
	    {\tt c-del} &    {\tt remeol} & Remove all text to the end-of-line \\
	    {\tt c-w} &      {\tt wordwrap toggle} & Switch {\tt WORDWRAP}-mode on and off \\
	    {\tt c-g} &      {\tt escimm (goto )} & Ask for a line and go to that line \\
	    {\tt c-l} &      {\verb/((^L))/} & Insert {\tt FF} \\
	    {\tt c-e} &      {\verb/((^[))/} & Insert {\tt ESC} \\
	    {\tt c-m} &      {\tt match} & Find matching parentheses \\
    \hline
	    {\tt a-n} &      {\tt next} & Find next occurrence of search-string \\
	    {\tt as-n} &     {\tt nextr} & Replace next occurrence of search-\\
		      & 	       & string by replace-string \\
	    {\tt a-p} &      {\tt prev} & Find previous occurrence of search-string \\
	    {\tt as-p} &     {\tt prevr} & Replace previous occurrence of search- \\
		      & 	       & string by replace-string \\
	    {\tt a-f} &      {\tt reformat} & Reformat paragraph \\
	    {\tt a-i} &      {\tt indent b .t} & Indent block by one {\tt TAB} to the right \\
	    {\tt a-I} &      {\tt indent b -.t} & Indent block by one {\tt TAB} to the left \\
	    {\tt a-b} &      {\tt block} & Mark block (like {\tt A-b} \\
	    {\tt a-c} &      {\tt bcopy} & Copy block to cursor \\
	    {\tt a-d} &      {\tt bdelete} & Delete block \\
	    {\tt a-m} &      {\tt bmove} & Move block to cursor \\
	    {\tt a-s} &      {\tt barpsave} & Save block to file via file-requester \\
    \hline
	    {\tt L-lmb} &    {\tt tomouse} & Jump to mouse \\
	    {\tt L-mmove} &  {\tt tomouse} & Follow mouse \\
	    {\tt M-mmb} &    {\tt clipins} & Insert contents of Clip \\
    \hline
    \end{tabular}
    \end{table}

\end{document}

%\begin{itemize}
%\item {\tt }
%\end{itemize}

